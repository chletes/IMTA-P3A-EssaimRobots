\section{Introduction}

Ce projet s’inscrit dans le contexte de la localisation des robots mobiles évoluant en essaims, i.e. devant potentiellement interagir et/ou communiquer entre eux, a minima pour éviter toutes collisions. Dans ce contexte, la connaissance de leur propres positions dans un repère absolu est primordiale. L’objectif est de mettre en place une installation « low cost » d’essaims de deux robots, dans une salle dédiée pourvu d’un système de « GPS d’intérieur ». 

\noindent Le système de localisation absolu sera de la marque Marvelmind \cite{marvelmind-url}. A l’aide de balises fixes et mobiles (embarquées sur les robots à localiser), d'un modem branché à l'ordinateur et d'un logiciel \textit{(Dashboard)}, le système permet d’avoir une mesure absolue relativement précise ($\pm$ 2 cm) de la position et de l’orientation des balises (en soi, des robots). 

\noindent Les robots mobiles sont les robots Pololu Zumo 32U4 \cite{pololu-url},  pourvus d’une librairie d’utilisation facile à prendre en main, car compatible avec l’environnement Arduino. \\[1pt]


Une des étapes primordiales dans notre projet est la formulation du problème de commande. Dans ce qui suit, nous allons voir en plus de détails le système que nous proposons pour résoudre notre problème de commande de l'essaim de robot, ainsi que les choix qui ont été fait. Ensuite, nous traitons le modèle cinématique de notre problème de commande. 