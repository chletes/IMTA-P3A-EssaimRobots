\section{Introduction}

Ce projet s’inscrit dans le contexte de la localisation des robots mobiles évoluant en essaims, i.e. devant potentiellement interagir et/ou communiquer entre eux, a minima pour éviter toutes collisions. Dans ce contexte, la connaissance de leur propres positions dans un repère absolu est primordiale. Ce projet possède deux objectifs. Le premier objectif est de mettre en place une installation \og low cost \fg{} d’essaims de deux robots, dans une salle dédiée pourvu d’un système de \og GPS d’intérieur \fg{}. Et le deuxième objectif est de formuler le problème de commande des deux robots pour contrôler l'interdistance entre les robots tout le long d'une trajectoire donnée.

\noindent Les robots mobiles sont les robots Pololu Zumo 32U4 \cite{pololu-url},  pourvus d’une librairie d’utilisation facile à prendre en main, car compatible avec l’environnement Arduino. \\[1pt]


Une des étapes primordiales dans notre projet est la formulation du problème de commande. Dans ce qui suit, nous allons voir en plus de détails le système que nous proposons pour résoudre notre problème de commande de l'essaim de robot, ainsi que les choix qui ont été fait. Ensuite, nous traitons le modèle cinématique de notre problème de commande. 