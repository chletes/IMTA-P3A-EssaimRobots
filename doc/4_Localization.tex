\section{Localisation}

La position du robot peut être représenté par l'équation (\ref{eq:position}). À chaque instant, la position peut être estimée à partir d'une position connue en intégrant le mouvement (en additionnant les distances de déplacement incrémentielles).

Pour un système discret avec un intervalle d'échantillonnage fixe $\Delta t$, les distances de déplacement incrémentielles $\left ( \Delta x, \Delta y, \Delta \theta \right )$, c'est-à-dire le chemin parcouru dans le dernier intervalle d'échantillonnage, sont: 
\begin{equation}
    \Delta x = \Delta s \cdot \cos \left ( \theta + \nicefrac{\Delta \theta}{2} \right )
\end{equation}

\begin{equation}
    \Delta y = \Delta s \cdot \sin \left ( \theta + \nicefrac{\Delta \theta}{2} \right )
\end{equation}

\begin{equation}
    \Delta \theta = \frac{\Delta s_r - \Delta s_l}{b}
\end{equation}

\begin{equation}
    \Delta s = \frac{\Delta s_r + \Delta s_l}{2}
\end{equation}

où: 
\begin{itemize}
    \item $\Delta s_r$ et $\Delta s_l$ sont les distances parcourues par la roue droite et la roue gauche, respectivement,
    \item $b$ est distance entre les deux roues du robot.
\end{itemize}


De ce fait, la position mise à jour peut être calculé grâce à:

\begin{equation}
    p' = p + \left [
\begin{array}{l}
     \Delta x  \\
     \Delta y  \\
     \Delta \theta
\end{array}
    \right ] = \left [
\begin{array}{l}
     x  \\
     y  \\
     \theta
\end{array}
    \right ] + \left [
\begin{array}{c}
     \frac{\Delta s_r + \Delta s_l}{2} \cdot \cos \left ( \theta + \frac{\Delta s_r - \Delta s_l}{2\cdot b} \right )  \\
     \frac{\Delta s_r + \Delta s_l}{2} \cdot \sin \left ( \theta + \frac{\Delta s_r - \Delta s_l}{2\cdot b} \right )  \\
     \frac{\Delta s_r - \Delta s_l}{b}
\end{array}
    \right ]
\end{equation}